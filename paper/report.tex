\documentclass[11pt,a4paper]{article}
\usepackage[utf8x]{inputenc}
\usepackage{ucs}
\usepackage{amsmath}
\usepackage{amsfonts}
\usepackage{amssymb}
\usepackage{fancyvrb}
\usepackage{url}
\author{HU, Pili and LI, Yichao}
\title{Community Detection on 2-Hop Topology}

\newcommand{\question}{\textbf{---NEED-REVIEW-HERE---}}

\begin{document}

\maketitle

\begin{abstract}
	Community detection in social networks has drawn great interest
	in recent decades, especially inspired by the emerge of wide 
	online social network sites. 

	This document serves as the report of CSCI5180 course 
	project\cite{csci5180_lecture,csci5180_tutorial}. 
	For more information, codes, and data, please refer to 
	our open source repository\cite{hu2011-cd2hop}. 
	 \question
\end{abstract}

\pagebreak
\tableofcontents
\pagebreak

\section{Introduction}

Community detection has drawn great interest in the near decade. 
The explosion of online Social Network Sites make the study more 
practical but challenging. Traditional methodology aims at 
optimizing certain global metric. The resultant graph partition 
can thus be regarded as communities. The motivation of detecting 
communities on 2-hop topology comes in two folds: 
\begin{enumerate}
	\item \textbf{Data Availability}. The global topology of 
	those SNS is kept confidential. However, users are allowed 
	to view their buddies profile by default. This gives us the 
	chance to construct a 2-hop subgraph from the whole graph. 
	Take a step further, outsiders can always observe some forwarding 
	sequences of tweet/status. This helps to reconstruct a 
	subgraph more than 2-hop from the observer. The methodology 
	developed here can thus be extended to such situations. 
	\item \textbf{Computation Availability}. Although those SNS 
	providers have access to the global topology, it is computationally
	intractable to perform some well-developed global optimization. 
	Local heuristics are widely used in many applications. 
\end{enumerate}

\section{Data Preparation and Preprocessing}

\subsection{Data Source}

The data source used in this paper is "renren.com", which is the 
most popular SNS in mainland China. At the mean while, "renren.com" 
has the following good characteristics:
\begin{itemize}
	\item Web UI of "3g.renren.com" is very clean. Cawling through
	this interface doesn't require any API calls.  
	\item Renren is a real name dense graph connected mainly through 
	real communities. Intuitively speaking, 
	people who are close to the observer should be observed within 
	this 2-hop topology with high probability. 
	\item Institution label for all nodes in the observed subgraph
	is available on "renren.com". It makes model validation easier
	and automated. If the model is proven effective, we can adapt 
	it to other SNS's. 
\end{itemize}



\section{Statistics}

\section{Feature Selection}

\subsection{Random Walk Based Techniques}


		\begin{itemize}
			\item PageRank with escape probability:
			\begin{equation}
				\overrightarrow{v} = (1-\alpha)P^{\rm T}\overrightarrow{v}
				+ \frac{\alpha}{n}\overrightarrow{1}
			\end{equation}
			Our graph is rooted and thus biased at the observer. So the 
			traditional rank output by this algorithm can be a proxity 
			measurement between other nodes and observer. 
			\item Personalized PageRank:
			\begin{equation}
				\overrightarrow{v} = (1-\alpha)P^{\rm T}\overrightarrow{v}
				+ \alpha\frac{\overrightarrow{b}}{||\overrightarrow{b}||_1}
			\end{equation}
			With different escape vector $\overrightarrow{b}$, we can reach 
			different goal:
				\begin{itemize}
					\item Unsupervised learning. Denote the subscript of observer
					as $o$, then:
					\begin{equation}
						b_i=\lbrace{\begin{tabular}{cc}
							1 & $i=o$ \\
							0 & $i\neq o$ 
						\end{tabular}}
					\end{equation}
					\item Semi-supervised learning. Denote the labeled set as 
					$L$ and their label as $l_i$. The personalized escape 
					vector can be:
					\begin{equation}
						b_i=\lbrace{\begin{tabular}{cc}
							$l_i$ & $i \in L $ \\
							$0$ & $i\notin L $ 
						\end{tabular}}
					\end{equation}
				\end{itemize}

		\end{itemize}  
		
\subsection{Simple Proximity Measures}

		\begin{itemize}

			\item Common Neighbours:
			\begin{equation}
				Common(i,j) = | N(i) \cap N(j) |
			\end{equation}
			\item Adamic/Adar score:
			\begin{equation}
				Adamic/Adar(i,j) = \sum_{k \in N(i) \cap N(j) }{\frac{1}{\log{|N(k)|}}}
			\end{equation}
			Adamic/Adar can be seen as an extension of simple common neighbours, which 
			take the neighbours degree into consideration. 
			\item Jaccards coefficient:
			\begin{equation}
				J(i,j)=\frac{|N(i) \cap N(j) |}{|N(i) \cup N(j) |}
			\end{equation}
			This coefficient can be seen as another extension of common neighbours. 
			It is effective when the graph is sparse. 
		\end{itemize}

\section{Single Feature Evalutaion}

\subsection{Quality Functions}



Evaluate some popular quality functions:
				\begin{itemize}
					\item Normalized cut:
					\begin{equation}
						Ncut(S)=\frac{\sum_{i \in S, j \in \overline{S}}{A(i,j)}}
						{\sum_{i \in S}{d(i)}}
						+ \frac{\sum_{i \in S, j \in \overline{S}}{A(i,j)}}
						{\sum_{j \in \overline{S}}{d(j)}}						
					\end{equation}
					\item Conductance:
					\begin{equation}
						Conductance(S)=\frac{\sum_{i \in S, j \in \overline{S}}{A(i,j)}}
						{\min \{ \sum_{i \in S}{d(i)}, \sum_{j \in \overline{S}}{d(j)}\}}				
					\end{equation}
					\item Modularity:
					\begin{equation}
						Q=\sum_{i=1}^{k}{\left[ 
						\frac{A(V_i,V_i)}{m} 
						-\left( \frac{d(V_i)}{2m}\right)^2
						\right]}
					\end{equation}
				\end{itemize}
				Those quality functions are popular among different research 
				groups. They also capture different characteristics of graphs. 
				In this project, we'll check if these global metric can 
				be used in our extreme local case. 


\subsection{Quality Functions}

Draw the Receiver Operating Curve by varying proximity threshold. 
			(Different TP and FP rate)
			

\question


\section{Model Training and Testing}


\section{Conclusion}


\section{Future Work}

As is in our proposal \cite{hu2011-cd2hop}, there are other metrics
we want to try. Due to project time and report space, we only name 
a few in this section:
\begin{itemize}
	\item Katz Score:
			\begin{equation}
				Katz(i,j)=\sum_{l=1}^{\infty}{\beta^lA^l(i,j)}
			\end{equation}
			The matrix series results in 
			\begin{equation}
				Katz = (I-\beta A)^{-1} - I
			\end{equation}
			By tuning $\beta$, this score can measure number of paths between 
			two nodes, with preferred length range. 
	\item Simrank:
			\begin{equation}
				s(a,b)=\frac{\gamma}{|I(a)||I(b)|}
				\sum_{i \in I(a), j \in I(b)} s(i,j)
			\end{equation}
			This metric measures the expectation of $\gamma^l$, where $l$ equals 
			the time when two random walkers start from $a$ and $b$ meet. By proper 
			matrix construction, Simrank can be solved as an eigenvalue problem. 
\end{itemize}
For more information, interested readers are recommended to \cite{aggarwal2011social}. 



\section*{Acknowledgements}


%begin================bibliography======================

%this seems impossible to merge automatically
%at the same time, I don't want to construct .bib databse
%for all these temporary entries. 
%
%my solution is to extract bibentries and paste them in 
%the following {thebibliography} section
%
%
%%\bibliographystyle{plain}
%%\bibliography{mylibrary}

%%%%GENBIBSTRING%%%%

\begin{thebibliography}{99}

%cut and paste section 
%generate by 'make bib' from my bib database
\bibitem{hu2011-cd2hop}
Hu~Pili and Li~Yichao.
\newblock Community detection on 2-hop topology.
\newblock GitHub, https://github.com/Czlyc/2C-Web-Research, 12 2011.
\newblock course project of CUHK/CSCI5180.

\bibitem{aggarwal2011social}
C.C. Aggarwal.
\newblock {\em Social network data analytics}.
\newblock Springer-Verlag New York Inc, 2011.


%temporary section
	\bibitem{csci5180_lecture} CSCI5180, Data Mining Lecture Notes, 
		\url{http://www.cse.cuhk.edu.hk/~lwchan/teaching/csc5180.html}
	\bibitem{csci5180_tutorial} CSCI5180, Data Mining Tutorial Notes. 
		(available in Moodle. Please contact the teacher/TA if you have 
		interest)
\end{thebibliography}

%end================bibliography======================


\section*{Appendix}


\section*{Declaration}

Meta tools like PageRank, Adamic/Adar, etc, are learned from the book
\textit{Social Network Data Analytics}. For original sources of these 
algorithms, please refer to the bibliography pages of that book.

The analysis/adaptation/extension/application of these tools and 
the use in community detection on 2-hop topology mentioned 
in this document is originality. Please refer to our open source 
repository for citation issues.

\end{document}