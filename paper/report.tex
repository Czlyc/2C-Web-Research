\documentclass[11pt,a4paper]{article}
\usepackage[utf8x]{inputenc}
\usepackage{ucs}
\usepackage{amsmath}
\usepackage{amsfonts}
\usepackage{amssymb}
\usepackage{fancyvrb}
\usepackage{url}
\author{HU, Pili and LI, Yichao}
\title{Community Detection on 2-Hop Topology}

\newcommand{\question}{\textbf{---NEED-REVIEW-HERE---}}

\begin{document}

\maketitle

\begin{abstract}
	Community detection in social networks has drawn great interest
	in recent decades, especially inspired by the emerge of wide 
	online social network sites. In this paper, we study the two-hop 
	topology from several observers in the "renren.com" network. 
	Some important statistics are calculated first, along with 
	some visualization. Then we compute mining(results)
	 \question

	This document serves as the report of CSCI5180 course 
	project\cite{csci5180_lecture,csci5180_tutorial}. 
	For more information, codes, and data, please refer to 
	our open source repository\cite{hu2011-cd2hop}. 
	
\end{abstract}

\pagebreak
\tableofcontents
\pagebreak

\section{Introduction}

Community detection has drawn great interest in the near decade. 
The explosion of online Social Network Sites make the study more 
practical but challenging. Traditional methodology aims at 
optimizing certain global metric. The resultant graph partition 
can thus be regarded as communities. The motivation of detecting 
communities on 2-hop topology comes in two folds: 
\begin{enumerate}
	\item \textbf{Data Availability}. The global topology of 
	those SNS is kept confidential. However, users are allowed 
	to view their buddies profile by default. This gives us the 
	chance to construct a 2-hop subgraph from the whole graph. 
	Take a step further, outsiders can always observe some forwarding 
	sequences of tweet/status. This helps to reconstruct a 
	subgraph more than 2-hop from the observer. The methodology 
	developed here can thus be extended to such situations. 
	\item \textbf{Computation Availability}. Although those SNS 
	providers have access to the global topology, it is computationally
	intractable to perform some well-developed global optimization. 
	Local heuristics are widely used in many applications. 
\end{enumerate}

\section{Data Preparation and Preprocessing}

\subsection{Data Source}

The data source used in this paper is "renren.com", which is the 
most popular SNS in mainland China. At the mean while, "renren.com" 
has the following good characteristics:
\begin{itemize}
	\item Web UI of "3g.renren.com" is very clean. Cawling through
	this interface doesn't require any API calls.  
	\item Renren is a real name dense graph connected mainly through 
	real communities. Intuitively speaking, 
	people who are close to the observer should be observed within 
	this 2-hop topology with high probability. 
	\item Institution label for all nodes in the observed subgraph
	is available on "renren.com". It makes model validation easier
	and automated. If the model is proven effective, we can adapt 
	it to other SNS's. 
\end{itemize}



\section{Statistics}

\section{Feature Selection}

\subsection{Random Walk Based Techniques}

The rationale behind random walk is: the nearer a node
is to observer, it can be reached with a higher probability by a random 
	walker starting from observer. PageRank is one of the most classical work 
	in this field.

We shortlist three variations\cite{aggarwal2011social} of the original 
PageRank, together with the explanation of using them in our context:
		\begin{itemize}
			\item PageRank with escape probability:
			\begin{equation}
				\overrightarrow{v} = \alpha A^{\rm T}\overrightarrow{v}
				+ (1-\alpha)\frac{\overrightarrow{1}}{n}
				\label{eq:pr}
			\end{equation}
			where $A$ is the transition matrix, typically the adjacency matrix,
			and $\alpha$ is the transfer ratio. $ \overrightarrow{v} $ is 
			the resultant PageRank value. 
			
			In our graph, the link is very sparse for edge node, which 
			represents the friend of observer's friend. They provide 
			many dangling links, which influences the outcome of 
			original PageRank algorithm. These nodes may appear as probability
			sinks. By introducing in escape probability, such sinks can be eliminated. 
			
			Our graph is rooted and thus biased at the observer. So the
			rank output by this algorithm can be a proximity 
			measure between other nodes and observer. 
			
			\item Personalized PageRank:
			\begin{equation}
				\overrightarrow{v} = \alpha A^{\rm T}\overrightarrow{v}
				+ (1-\alpha)\frac{\overrightarrow{b}}{||\overrightarrow{b}||_1}
				\label{eq:ppr}
			\end{equation}
			The difference from eqn(\ref{eq:pr}) is that, the all one's column vector 
			$\overrightarrow{1}$ is 
			substituted by a more general weight vector, also called the 
			escape vector. This vector describes where and by what probability 
			the random walker will escape to. 		
			
			With different escape vector $\overrightarrow{b}$, we can reach 
			different goal:
				\begin{itemize}
					\item Unsupervised learning. Denote the subscript of observer
					as $o$, then:
					\begin{equation}
						b_i=\lbrace{\begin{tabular}{cc}
							1 & $i=o$ \\
							0 & $i\neq o$ 
						\end{tabular}}
					\end{equation}
					That means the random walker will always escape to the root node. 
					Then the output can measure the proximity between root and other nodes. 
					\item Semi-supervised learning. Denote the labeled set as 
					$L$ and their label as $l_i$. The personalized escape 
					vector can be:
					\begin{equation}
						b_i=\lbrace{\begin{tabular}{cc}
							$l_i$ & $i \in L $ \\
							$0$ & $i\notin L $ 
						\end{tabular}}
					\end{equation}
					Typically, the miner can choose a subset of the nodes, which are
					already known to share the same label with the observer. We assign
					those nodes in set $L$ some positive weights. Then the random walker 
					can escape to this set with different probability. The rantionale is, 
					if one node is in our community, it is probable to be reached 
					from some already known members.  
				\end{itemize}

		\end{itemize}  
		
\subsection{Simple Proximity Measures}

Yet random walk based techniques are good proximity measures in many context, 
there exists some simple but effective measures. We shortlist the following 
metrics, together with explanation of using them in our context:
	\begin{itemize}
		\item Common Neighbours:
		\begin{equation}
			Common(i,j) = | N(i) \cap N(j) |
			\label{eq:common}
		\end{equation}
		It is straight forward that if node $i$ shares more common neighbours 
		with observer $o$, it is closer to observer. 
		
		\item Adamic/Adar score:
%change the equation back
%no need to do "1" opeartion, since this divide-by-zero flaw results
%from the situation where i==j, and one of the neighbours only have 
%degree 1. 
%=== 
%		\begin{equation}
%			Adamic/Adar(i,j) = \sum_{k \in N(i) \cap N(j) }{\frac{1}{\log{|N(k)|+1}}}
%			\label{eq:adar}
%		\end{equation}
%===
		\begin{equation}
			Adamic/Adar(i,j) = \sum_{k \in N(i) \cap N(j) }{\frac{1}{\log{|N(k)|}}}
			\label{eq:adar}
		\end{equation}
		Adamic/Adar can be seen as an extension of simple common neighbours, which 
		take the neighbour's degree into consideration. The contribution from 
%		each common neighbour $k$ is weighted as $\frac{1}{\log{|N(k)|+1}}$. 
		each common neighbour $k$ is weighted as $\frac{1}{\log{|N(k)|}}$. 
		Higher degree nodes may stant for public pages, or very well-known people 
		in the network. Sharing such kind of common neighbour is a much weaker 
		evidence that two people are in the same community. Thus the it is given 
		a lower weight. The use of log scale stems from previous statistical studies, 
		that node ranking tends to be Zipf distributed. \cite{breslau1999web-zipf}
		
		\item Jaccard's coefficient:
		\begin{equation}
			J(i,j)=\frac{|N(i) \cap N(j) |}{|N(i) \cup N(j) |}
			\label{jaccards}
		\end{equation}
		This coefficient can be seen as another extension of common neighbours. 
		It is effective when the graph is sparse. The numerator calculates the 
		common neighbours of two nodes. The denominator takes their own size
		into consideration. It's reasonable that two nodes with higher degree 
		will have more common neighbours. The denominator act as a normalizer, 
		which makes the output of different node pairs relative comparable. 
	\end{itemize}

\section{Single Feature Evalutaion}

\subsection{Quality Functions}

After we label the community of all nodes, we can calculate some quality 
functions to indicate how well the community is detected. Different researchers
have different preference of quality functions, to name a few:
	\begin{itemize}
		\item Normalized cut:
		\begin{equation}
			Ncut(S)=\frac{\sum_{i \in S, j \in \overline{S}}{A(i,j)}}
			{\sum_{i \in S}{d(i)}}
			+ \frac{\sum_{i \in S, j \in \overline{S}}{A(i,j)}}
			{\sum_{j \in \overline{S}}{d(j)}}					
			\label{eq:ncut}	
		\end{equation}
		where $S$ is the set of one community, 
		and $d(.)$ is the digree of a node, which is defined as 
		$d(i)=\sum_j{A_{ij}}$.
		
		The smaller the value, the better community detection quality. 
		Two numerators measure the number of inter community links, 
		and the value is normalized by the number of intra community links. 
		\item Conductance:
		\begin{equation}
			Conductance(S)=\frac{\sum_{i \in S, j \in \overline{S}}{A(i,j)}}
			{\min \{ \sum_{i \in S}{d(i)}, \sum_{j \in \overline{S}}{d(j)}\}}	
			\label{eq:conductance}			
		\end{equation}
		This quality function is positive related with eqn(\ref{eq:ncut}). 
		When community sizes are highly skewed, conductance may provide 
		better evaluation. 
		\item Modularity:
		\begin{equation}
			Q=\sum_{i=1}^{K}{\left[ 
			\frac{A(V_i,V_i)}{m} 
			-\left( \frac{d(V_i)}{2m}\right)^2
			\right]}
			\label{eq:modularity}
		\end{equation}
		where 
		\begin{equation}
			A(V_i,V_j)= \sum_{u \in V_i,v \in V_j}{A_{uv}}
		\end{equation}
		and 
		\begin{equation}
			d(V_i)=\sum_{u \in V_i}{d_u}
		\end{equation}
		$K$ is the number of communities, which is 2 in our case
		(we only distinguish whether the node is in the same community
		with observer or not).
		
		The rationale behind modularity is the comparison with a random 
		graph. The father resultant community is from a random graph, 
		the better the quality. One of the advantages of modularity is 
		that it is independent of the number of communities the graph is
		divided into\cite{aggarwal2011social}. 
	\end{itemize}
	
	Those quality functions are popular among different research 
	groups. They also capture different characteristics of graphs. 
	In this project, we'll check if these global metric can 
	be used to evaluate our extreme local case. 


\subsection{Receiver Operating Characteristics}

Draw the Receiver Operating Curve by varying proximity threshold. 
			(Different TP and FP rate)




\question


\section{Model Training and Testing}


\section{Conclusion}


\section{Future Work}

As is in our proposal \cite{hu2011-cd2hop}, there are other metrics
we want to try. Due to project time and report space, we only name 
a few in this section:
\begin{itemize}
	\item Katz Score:
			\begin{equation}
				Katz(i,j)=\sum_{l=1}^{\infty}{\beta^lA^l(i,j)}
			\end{equation}
			The matrix series results in 
			\begin{equation}
				Katz = (I-\beta A)^{-1} - I
			\end{equation}
			By tuning $\beta$, this score can measure number of paths between 
			two nodes, with preferred length range. 
	\item Simrank:
			\begin{equation}
				s(a,b)=\frac{\gamma}{|I(a)||I(b)|}
				\sum_{i \in I(a), j \in I(b)} s(i,j)
			\end{equation}
			This metric measures the expectation of $\gamma^l$, where $l$ equals 
			the time when two random walkers start from $a$ and $b$ meet. By proper 
			matrix construction, Simrank can be solved as an eigenvalue problem. 
\end{itemize}
For more information, interested readers are recommended to \cite{aggarwal2011social}. 



\section*{Acknowledgements}

The authors would like to thank the following people for their contribution of 
personal data: 
Bo WANG, Hongshu LIAO, Huan CHEN, Shouxi LUO, 
Xiaoyu LUO, Xinyue ZHENG, Yan WANG, Yi LUO. 
The authors appreciate brain storm discussion with professor Wing Lau 
and Deyi Sun. 
The authors would like to thank course instructors and TAs of CSCI5180. 


%begin================bibliography======================

%this seems impossible to merge automatically
%at the same time, I don't want to construct .bib databse
%for all these temporary entries. 
%
%my solution is to extract bibentries and paste them in 
%the following {thebibliography} section
%
%
%%\bibliographystyle{plain}
%%\bibliography{mylibrary}

%%%%GENBIBSTRING%%%%

\begin{thebibliography}{99}

%cut and paste section 
%generate by 'make bib' from my bib database
\bibitem{hu2011-cd2hop}
Hu~Pili and Li~Yichao.
\newblock Community detection on 2-hop topology.
\newblock GitHub, https://github.com/Czlyc/2C-Web-Research, 12 2011.
\newblock course project of CUHK/CSCI5180.

\bibitem{aggarwal2011social}
C.C. Aggarwal.
\newblock {\em Social network data analytics}.
\newblock Springer-Verlag New York Inc, 2011.

\bibitem{breslau1999web-zipf}
L.~Breslau, P.~Cao, L.~Fan, G.~Phillips, and S.~Shenker.
\newblock Web caching and zipf-like distributions: Evidence and implications.
\newblock In {\em INFOCOM'99. Eighteenth Annual Joint Conference of the IEEE
  Computer and Communications Societies. Proceedings. IEEE}, volume~1, pages
  126--134. IEEE, 1999.


%temporary section
	\bibitem{csci5180_lecture} CSCI5180, Data Mining Lecture Notes, 
		\url{http://www.cse.cuhk.edu.hk/~lwchan/teaching/csc5180.html}
	\bibitem{csci5180_tutorial} CSCI5180, Data Mining Tutorial Notes. 
		(available in Moodle. Please contact the teacher/TA if you have 
		interest)
	\bibitem{wiki_roc} Receiver Operating Characteristics, 
		\url{http://en.wikipedia.org/wiki/Receiver_operating_characteristic}
\end{thebibliography}

%end================bibliography======================


\section*{Appendix}


\section*{Declaration}

Meta tools like PageRank, Adamic/Adar, etc, are learned from the book
\textit{Social Network Data Analytics}. For original sources of these 
algorithms, please refer to the bibliography pages of that book.

The analysis/adaptation/extension/application of these tools and 
the use in community detection on 2-hop topology mentioned 
in this document is originality. Please refer to our open source 
repository for citation issues.

\end{document}
